\label{app:polynomials}Here we collect information regarding classical
orthogonal polynomials. Most of this can be found in e.g. {\cite{szego1975}},
{\cite{abramowitz1972}}, and {\cite{gautschi2004}}. Among all properties that
orthogonal polynomials satisfy, perhaps the most useful is the three-term
recurrence relation
\begin{eqnarray}
  p_{n + 1} & = & (x - a_n) p_n - b_n p_{n - 1}  \label{eq:opoly3term-monic}\\
  &  &  \nonumber\\
  \sqrt{b_{n + 1}} \tilde{p}_{n + 1} & = & (x - a_n) \tilde{p}_n - \sqrt{b_n} 
  \tilde{p}_{n - 1},  \label{eq:opoly3term-normal}
\end{eqnarray}
where $p_n$ denotes the monic polynomial, and $\tilde{p}_n$ denotes the
orthonormal polynomial.

\section{Jacobi Polynomials}

\label{app:polynomials-jacobi}The Jacobi polynomials $P_n^{(\alpha, \beta)}$
are polynomial solutions to the differential equation
\begin{equation}
  \label{eq:jacobi-differential-equation} \begin{array}{ll}
    (1 - r^2) \rho'' + \left[ \beta - \alpha - (\alpha + \beta + 2) r \right]
    \rho' + n (n + \alpha + \beta + 1) \rho = 0, & r \in [- 1, 1] .
  \end{array}
\end{equation}
In canonical Sturm-Liouville form, this reads
\begin{equation}
  \label{eq:slp-jacobi-canonical} - \frac{\mathd}{\mathd r}  \left[ (1 -
  r)^{\alpha + 1} (1 + r)^{\beta + 1} \rho' \right] - n (n + \alpha + \beta +
  1) (1 - r)^{\alpha} (1 + r)^{\beta} \rho = 0.
\end{equation}

One can read off the eigenvalues as $\lambda_n = n (n + \alpha +
\beta + 1)$.  One thus has that the Jacobi polynomials are orthogonal
under the weight $w_r^{(\alpha, \beta)}$. We restrict the parameters
$\alpha,\beta>-1$ to ensure integrability of the weight function and
thus existence of a family of orthogonal polynomials.


In the literature the common way to normalize Jacobi Polynomials is by the
characterization
\[ \hat{P}_n^{(\alpha, \beta)} (r = 1) = \left( \begin{array}{c}
     n + \alpha\\
     n
   \end{array} \right) = \frac{\Gamma (n + \alpha + 1)}{\Gamma (\alpha + 1)
   n!}, \]
where $\Gamma \left( \cdot \right)$ represents the Gamma function. For
reference, the norm $\left\| \hat{P}_n^{(\alpha, \beta)}
\right\|_{w_r^{(\alpha, \beta)}}^2 = \hat{h}_n^{(\alpha, \beta)}$ is defined
as
\[ h_n^{(\alpha, \beta)} = \frac{2^{\alpha + \beta + 1}}{2 n + \alpha + \beta
   + 1}  \frac{\Gamma (n + \alpha + 1) \Gamma (n + \beta + 1)}{\Gamma (n + 1)
   \Gamma (n + \alpha + \beta + 1)}, \]
These orthogonal polynomials are derived from the recurrence relation
\[ p_{n + 1} = ( \hat{a}_n x - \hat{b}_n) p_n - \hat{c}_n p_{n - 1}, \]
where the constants $\hat{a}_n$, $\hat{b}_n$, and $\hat{c}_n$ are given by
\[ \begin{array}{lll}
     \hat{a}^{(\alpha, \beta)}_n & = & \frac{(2 n + \alpha + \beta + 1) (2 n +
     \alpha + \beta + 2)}{2 (n + 1) (n + \alpha + \beta + 1)}\\
     &  & \\
     \hat{b}^{(\alpha, \beta)}_n & = & \frac{(\alpha^2 - \beta^2) (2 n +
     \alpha + \beta + 1)}{2 (n + 1) (2 n + \alpha + \beta) (n + \alpha + \beta
     + 1)}\\
     &  & \\
     \hat{c}^{(\alpha, \beta)}_n & = & \frac{(n + \alpha) (n + \beta) (2 n +
     \alpha + \beta + 2)}{(n + 1) (n + \alpha + \beta + 1) (2 n + \alpha +
     \beta)} .
   \end{array} \]
These polynomials have leading coefficient equal to
\[ \hat{k}_n^{(\alpha, \beta)} = \left\{ \begin{array}{ll}
     1, & n = 0\\
     & \\
     \frac{\Gamma (2 n + \alpha + \beta + 1)}{\Gamma (n + \alpha + \beta + 1)
     2^n n!}, & \tmop{else} .
   \end{array} \right. \]


We write the monic Jacobi polynomials as $P^{(\alpha, \beta)}_n (r)$ for any
$n \in \mathbbm{N}_0$. $P^{(\alpha, \beta)}_n (r)$
is a solution to (\ref{eq:slp-jacobi}) and for every $n, m \in \mathbbm{N}_0$
we have the orthogonality relation
\[ \int_{- 1}^1 P_m^{(\alpha, \beta)} (r) P_n^{(\alpha, \beta)} (r) (1 -
   r)^{\alpha} (1 + r)^{\beta} \mathd r = h_n^{(\alpha, \beta)} \delta_{m, n},
\]
where the monic polynomials have norm $\left\| P_n^{(\alpha, \beta)}
\right\|_{}^2 = h_n^{(\alpha, \beta)}$, which is given by


\begin{equation}
  \label{eq:monic-h} \begin{array}{lll}
    h_n^{(\alpha, \beta)} & = & \left\{ \begin{array}{ll}
      \frac{2^{\alpha + \beta + 1} \Gamma (\alpha + 1) \Gamma (\beta +
      1)}{\Gamma (\alpha + \beta + 2)}, & n = 0\\
      & \\
      \frac{2^{2 n + \alpha + \beta + 1} (n!) \Gamma (n + \alpha + \beta + 1)
      \Gamma (n + \alpha + 1) \Gamma (n + \beta + 1)}{\Gamma (2 n + \alpha +
      \beta + 1) \Gamma (2 n + \alpha + \beta + 2)}, & \tmop{else}
    \end{array} \right.
  \end{array}
\end{equation}
Furthermore, the leading and second coefficients for the monic Jacobi
polynomials are
\[ \begin{array}{lll}
     k_n^{(\alpha, \beta)} & = & 1\\
     &  & \\
     k_n^{(\alpha, \beta)'} & = & \frac{n (\alpha - \beta)}{2 n + \alpha +
     \beta} .
   \end{array} \]
The monic Jacobi polynomials satisfy various relations and can be expressed in
terms of the Hypergeometric function $_2 F_1$, and also satisfy a generalized
Rodrigues relation for all $m \leq n$:
\begin{equation}
  \label{eq:rodrigues} \begin{array}{l}
    \left( \begin{array}{c}
      2 n + \alpha + \beta\\
      n
    \end{array} \right) \omega^{(\alpha, \beta)} P_n^{(\alpha, \beta)} =\\
    \hspace{1.0cm} \frac{(- 1)^m}{2^{m - n} (n - 1) \cdots (n - m + 1)}
    \frac{\mathd^m}{\mathd x^m} \left[ \omega^{(\alpha + m, \beta + m)} P_{n -
    m}^{(\alpha + m, \beta + m)} \right] .
  \end{array}
\end{equation}


We shall often work with the normalized Jacobi polynomials,
$\tilde{P}^{(\alpha, \beta)}_n (r) = \frac{P^{(\alpha, \beta)}_n
(r)}{\sqrt{h^{(\alpha, \beta)}_n}}$, which are orthonormal with respect to the
associated weight. Naturally these polynomials have the $L^2$-normalization
constant $\tilde{h}_n^{(\alpha, \beta)} \equiv 1$.



For the Jacobi case, the recurrence coefficients in 
(\ref{eq:opoly3term-monic}-\ref{eq:opoly3term-normal}) used to generate the
monic and orthonormal polynomials are given by
\[ \begin{array}{lll}
     a^{(\alpha, \beta)}_n & = & \left\{ \begin{array}{ll}
       \frac{\beta - \alpha}{(\alpha + \beta + 2)}, & n = 0\\
       & \\
       \frac{\beta^2 - \alpha^2}{(2 n + \alpha + \beta) (2 n + \alpha + \beta
       + 2)}, & n > 0
     \end{array} \right.\\
     &  & \\
     b^{(\alpha, \beta)}_n & = & \left\{ \begin{array}{ll}
       \frac{2^{\alpha + \beta + 1} \Gamma (\alpha + 1) \Gamma (\beta +
       1)}{\Gamma (\alpha + \beta + 2)}, & n = 0\\
       & \\
       \frac{4 (1 + \alpha) (1 + \beta)}{(2 + \alpha + \beta)^2 (3 + \alpha +
       \beta)}, & n = 1\\
       & \\
       \frac{4 n (n + \alpha) (n + \beta) (n + \alpha + \beta)}{(2 n + \alpha
       + \beta)^2 (2 n + \alpha + \beta + 1) (2 n + \alpha + \beta - 1)}, & n
       > 1
     \end{array} \right.
   \end{array} \]


Define $\theta$ as $r = \cos \theta$. There are some particular choices for
$\alpha$ and $\beta$ which are of special note. These are the Chebyshev
polynomials of the first, second, third, and fourth kinds, respectively:


\begin{eqnarray}
  &  & t_n \tilde{P}_n^{(- 1 / 2, - 1 / 2)} (r) = T_n (r) = \cos (n \theta) 
  \label{eq:cheb1-def}\\
  &  &  \nonumber\\
  &  & u_n \tilde{P}_n^{(+ 1 / 2, + 1 / 2)} (r) = U_n (r) = \frac{\sin (n +
  1) \theta}{\sin \theta}  \label{eq:cheb2-def}\\
  &  &  \nonumber\\
  &  & v_n \tilde{P}_n^{(- 1 / 2, + 1 / 2)} (r) = V_n (r) = \frac{\cos \left(
  n + \frac{1}{2} \right) \theta}{\cos \frac{1}{2} \theta} 
  \label{eq:cheb3-def}\\
  &  &  \nonumber\\
  &  & w_n \tilde{P}_n^{(+ 1 / 2, - 1 / 2)} (r) = W_n (r) = \frac{\sin \left(
  n + \frac{1}{2} \right) \theta}{\sin \frac{1}{2} \theta} 
  \label{eq:cheb4-def}
\end{eqnarray}
where $w_n = v_n = \sqrt{\pi}$; $u_n = \sqrt{\frac{\pi}{2}}$; and $t_n =
\sqrt{\frac{\pi}{2}}$ for $n > 0$, and $t_0 = \sqrt{\pi}$.



In Chapter \ref{sec:jacobifft} we outlined a method to use the Fast Fourier
Transform for computing spectral coefficients of Jacobi Polynomial transforms.
The relations requisite for such an algorithm rely on expressions given in
(\ref{eq:jdemotiona})-(\ref{eq:jpromotionb}); these equations
contain certain constants. These constants are derived in 
(\ref{eq:jacform-gammatildedef}) and (\ref{eq:jacform-deltatildedef}) below;
for reference we reproduce them here:
\begin{eqnarray}
  \mu_{n, 0}^{(\alpha, \beta)} & = & \sqrt{\frac{2 (n + \alpha) (n + \alpha +
  \beta)}{(2 n + \alpha + \beta) (2 n + \alpha + \beta + 1)}} 
  \label{eq:mu0}\\
  &  &  \nonumber\\
  \mu_{n, 1}^{(\alpha, \beta)} & = & \sqrt{\frac{2 (n + 1) (n + \beta + 1)}{(2
  n + \alpha + \beta + 1) (2 n + \alpha + \beta + 2)}}  \label{eq:mu1}\\
  &  &  \nonumber\\
  \nu_{n, 0}^{(\alpha, \beta)} & = & \sqrt{\frac{2 (n + \alpha + 1) (n +
  \alpha + \beta + 1)}{(2 n + \alpha + \beta + 1) (2 n + \alpha + \beta + 2)}}
  \label{eq:nu0}\\
  &  &  \nonumber\\
  \nu_{n, - 1}^{(\alpha, \beta)} & = & \sqrt{\frac{2 n (n + \beta)}{(2 n +
  \alpha + \beta) (2 n + \alpha + \beta + 1)}}  \label{eq:nu1}
\end{eqnarray}


Using the normalization constant for the monic Jacobi polynomials, we derive,
for $n > 0$:
\[ \begin{array}{lll}
     \frac{h_n^{(\alpha - 1, \beta)}}{h_n^{(\alpha, \beta)}} & = & \frac{(2 n
     + \alpha + \beta) (2 n + \alpha + \beta + 1)}{2 (n + \alpha) (n + \alpha
     + \beta)}\\
     &  & \\
     \frac{h_n^{(\alpha, \beta - 1)}}{h_n^{(\alpha, \beta)}} & = & \frac{(2 n
     + \alpha + \beta) (2 n + \alpha + \beta + 1)}{2 (n + \beta) (n + \alpha +
     \beta)}\\
     &  & \\
     \frac{h_{n - 1}^{(\alpha, \beta)}}{h_n^{(\alpha, \beta)}} & = & \frac{(2
     n + \alpha + \beta - 1) (2 n + \alpha + \beta)^2 (2 n + \alpha + \beta +
     1)}{4 n (n + \alpha) (n + \beta) (n + \alpha + \beta)}
   \end{array} \]
We claim that with these three formulas, we have obtained the recurrence
constants for a great variety of nontrival relations. To give examples of how
we can derive other types of formulas from these three, we have e.g.
\[ \begin{array}{lllll}
     \frac{h^{(\alpha - 1, \beta)}_{n + 1}}{h_n^{(\alpha, \beta)}} & = &
     \frac{h_{n + 1}^{(\alpha - 1, \beta)}}{h_{n + 1}^{(\alpha, \beta)}} 
     \frac{h_{n + 1}^{(\alpha, \beta)}}{h_n^{(\alpha, \beta)}} & = & \frac{2
     (n + 1) (n + \beta + 1)}{(2 n + \alpha + \beta + 1) (2 n + \alpha + \beta
     + 2)}\\
     &  &  &  & \\
     \frac{h_{n + 1}^{(\alpha, \beta - 1)}}{h_n^{(\alpha, \beta)}} & = &
     \frac{h_{n + 1}^{(\alpha, \beta - 1)}}{h_{n + 1}^{(\alpha, \beta)}} 
     \frac{h_{n + 1}^{(\alpha, \beta)}}{h_n^{(\alpha, \beta)}} & = & \frac{2
     (n + 1) (n + \alpha + 1)}{(2 n + \alpha + \beta + 1) (2 n + \alpha +
     \beta + 2)} .
   \end{array} \]
These new derived formulas may not seem immediately useful, but such
expressions form the basis for all the recurrence formulas we now derive for
the $L^2$ normalized Jacobi polynomials from known properties of the monic
polynomials:


\begin{eqnarray}
  (1 - r) P_n^{(\alpha, \beta)} & = & \gamma_{n, 0}^{(\alpha, \beta)}
  P_n^{(\alpha - 1, \beta)} - \gamma_{n, 1}^{(\alpha, \beta)} P_{n +
  1}^{(\alpha - 1, \beta)}  \label{eq:jacform-ad}\\
  &  &  \nonumber\\
  (1 + r) P_n^{(\alpha, \beta)} & = & \gamma_{n, 0}^{(\beta, \alpha)}
  P_n^{(\alpha, \beta - 1)} + \gamma_{n, 1}^{(\beta, \alpha)} P_{n +
  1}^{(\alpha, \beta - 1)}  \label{eq:jacform-bd}\\
  &  &  \nonumber\\
  &  & \left. \begin{array}{lll}
    \gamma_{n, 0}^{(\alpha, \beta)} & = & \frac{2 (n + \alpha) (n + \alpha +
    \beta)}{(2 n + \alpha + \beta) (2 n + \alpha + \beta + 1)} =
    \frac{h_n^{(\alpha, \beta)}}{h_n^{(\alpha - 1, \beta)}}\\
    &  & \\
    \gamma_{n, 1}^{(\alpha, \beta)} & = & 1
  \end{array} \right\}  \label{eq:jacform-gammadef}\\
  &  &  \nonumber\\
  (1 - r) \tilde{P}_n^{(\alpha, \beta)} & = & \mu^{(\alpha, \beta)}_{n, 0} 
  \tilde{P}_n^{(\alpha - 1, \beta)} - \mu_{n, 1}^{(\alpha, \beta)} 
  \tilde{P}_{n + 1}^{(\alpha - 1, \beta)}  \label{eq:jacform-adn}\\
  &  &  \nonumber\\
  (1 + r) \tilde{P}_n^{(\alpha, \beta)} & = & \mu_{n, 0}^{(\beta, \alpha)}
  \tilde{P}_n^{(\alpha, \beta - 1)} + \mu^{(\beta, \alpha)}_{n, 1}
  \tilde{P}_{n + 1}^{(\alpha, \beta - 1)}  \label{eq:jacform-bdn}\\
  &  &  \nonumber\\
  &  & \left. \begin{array}{lll}
    \mu^{(\alpha, \beta)}_{n, 1} & = & \sqrt{\frac{2 (n + 1) (n + \beta +
    1)}{(2 n + \alpha + \beta + 1) (2 n + \alpha + \beta + 2)}} =
    \sqrt{\frac{h_{n + 1}^{(\alpha - 1, \beta)}}{h_n^{(\alpha, \beta)}}}\\
    &  & \\
    \mu^{(\alpha, \beta)}_{n, 0} & = & \sqrt{\frac{2 (n + \alpha) (n + \alpha
    + \beta)}{(2 n + \alpha + \beta) (2 n + \alpha + \beta + 1)}} =
    \sqrt{\frac{h_n^{(\alpha, \beta)}}{h_n^{(\alpha - 1, \beta)}}}
  \end{array} \right\}  \label{eq:jacform-gammatildedef}\\
\end{eqnarray}
\begin{eqnarray}
  P_n^{(\alpha, \beta)} & = & \delta_{n, 0}^{(\alpha, \beta)} P_n^{(\alpha +
  1, \beta)} - \delta_{n, - 1}^{(\alpha, \beta)} P_{n - 1}^{(\alpha + 1,
  \beta)}  \label{eq:jacform-ap}\\
  &  &  \nonumber\\
  P_n^{(\alpha, \beta)} & = & \delta_{n, 0}^{(\beta, \alpha)} P_n^{(\alpha,
  \beta + 1)} + \delta_{n, - 1}^{(\beta, \alpha)} P_{n - 1}^{(\alpha, \beta +
  1)}  \label{eq:jacform-bp}\\
  &  &  \nonumber\\
  &  & \left. \begin{array}{lll}
    \delta_{n, 0}^{(\alpha, \beta)} & = & 1\\
    &  & \\
    \delta_{n, - 1}^{(\alpha, \beta)} & = & \frac{2 n (n + \beta)}{(2 n +
    \alpha + \beta) (2 n + \alpha + \beta + 1)} = \frac{h_n^{(\alpha,
    \beta)}}{h_{n - 1}^{(\alpha + 1, \beta)}}
  \end{array} \right\}  \label{eq:jacform-deltadef}\\
  &  &  \nonumber\\
  \tilde{P}_n^{(\alpha, \beta)} & = & \nu^{(\alpha, \beta)}_{n, 0}
  \tilde{P}_n^{(\alpha + 1, \beta)} - \nu^{(\alpha, \beta)}_{n, - 1}
  \tilde{P}_{n - 1}^{(\alpha + 1, \beta)}  \label{eq:jacform-apn}\\
  &  &  \nonumber\\
  \tilde{P}_n^{(\alpha, \beta)} & = & \nu_{n, 0}^{(\beta, \alpha)} 
  \tilde{P}^{(\alpha, \beta + 1)}_n + \nu^{(\beta, \alpha)}_{n, - 1} 
  \tilde{P}^{(\alpha, \beta + 1)}_{n - 1}  \label{eq:jacform-bpn}\\
  &  &  \nonumber\\
  &  & \left. \begin{array}{lll}
    \nu^{(\alpha \beta)}_{n, 0} & = & \sqrt{\frac{2 (n + \alpha + 1) (n +
    \alpha + \beta + 1)}{(2 n + \alpha + \beta + 1) (2 n + \alpha + \beta +
    2)}} = \sqrt{\frac{h_n^{(\alpha + 1, \beta)}}{h_n^{(\alpha, \beta)}}}\\
    &  & \\
    \nu^{(\alpha, \beta)}_{n, - 1} & = & \sqrt{\frac{2 n (n + \beta)}{(2 n +
    \alpha + \beta) (2 n + \alpha + \beta + 1)}} = \sqrt{\frac{h_n^{(\alpha,
    \beta)}}{h_{n - 1}^{(\alpha + 1, \beta)}}}
  \end{array} \right\}  \label{eq:jacform-deltatildedef}
\end{eqnarray}


Putting these things together, we have
\begin{eqnarray}
  (1 - r^2) P_n^{(\alpha, \beta)} & = & \sum_{k = 0}^2 \varepsilon_{n,
  k}^{(\alpha, \beta)} P_{n + k}^{(\alpha - 1, \beta - 1)} 
  \label{eq:jacform-abd}\\
  &  &  \nonumber\\
  &  & \left. \begin{array}{lll}
    \varepsilon_{n, 1}^{(\alpha, \beta)} & = & \frac{2 (\alpha - \beta) (n +
    \alpha + \beta)}{(2 n + \alpha + \beta) (2 n + \alpha + \beta + 2)}\\
    &  & \\
    \varepsilon_{n, 0}^{(\alpha, \beta)} & = & \frac{4 (n + \alpha) (n +
    \beta) (n + \alpha + \beta - 1) (n + \alpha + \beta)}{(2 n + \alpha +
    \beta - 1) (2 n + \alpha + \beta)^2 (2 n + \alpha + \beta + 1)}\\
    &  & \\
    & = & \frac{h_n^{(\alpha, \beta)}}{h_n^{(\alpha - 1, \beta - 1)}}\\
    &  & \\
    \varepsilon^{(\alpha, \beta)}_{n, 2} & = & - 1
  \end{array} \right\}  \label{eq:jacform-epsilondef}\\
  &  &  \nonumber\\
  (1 - r^2) \tilde{P}^{(\alpha, \beta)}_n & = & \sum_{k = 0}^2
  \tilde{\varepsilon}_{n, k}^{(\alpha, \beta)}  \tilde{P}_{n + k}^{(\alpha -
  1, \beta - 1)}  \label{eq:jacform-abdn}\\
  &  &  \nonumber\\
  &  & \left. \begin{array}{lll}
    \tilde{\varepsilon}_{n, 0}^{(\alpha, \beta)} & = & \sqrt{\varepsilon_{n,
    0}^{(\alpha, \beta)}}\\
    &  & \\
    & = & \sqrt{\frac{4 (n + \alpha) (n + \beta) (n + \alpha + \beta - 1) (n
    + \alpha + \beta)}{(2 n + \alpha + \beta - 1) (2 n + \alpha + \beta)^2 (2
    n + \alpha + \beta + 1)}}\\
    &  & \\
    & = & \sqrt{\frac{h_n^{(\alpha, \beta)}}{h_n^{(\alpha - 1, \beta -
    1)}}}\\
    &  & \\
    \tilde{\varepsilon}^{(\alpha, \beta)}_{n, 1} & = & \varepsilon_{n,
    1}^{(\alpha, \beta)}  \sqrt{\frac{h_{n + 1}^{(\alpha - 1, \beta -
    1)}}{h_n^{(\alpha \beta)}}} = \frac{2 (\alpha - \beta) \sqrt{(n + 1) (n +
    \alpha + \beta)}}{(2 n + \alpha + \beta) (2 n + \alpha + \beta + 2)}\\
    &  & \\
    \tilde{\varepsilon}^{(\alpha, \beta)}_{n, 2} & = & -
    \sqrt{\frac{h^{(\alpha - 1, \beta - 1)}_{n + 2}}{h_n^{(\alpha, \beta)}}}\\
    &  & \\
    & = & - \sqrt{\frac{4 (n + 1) (n + 2) (n + \alpha + 1) (n + \beta +
    1)}{(2 n + \alpha + \beta + 1) (2 n + \alpha + \beta + 2)^2 (2 n + \alpha
    + \beta + 3)}}
  \end{array} \right\}  \label{eq:jacform-epsilontildedef}
\end{eqnarray}

\begin{eqnarray}
  P_n^{(\alpha, \beta)} & = & \sum_{k = - 2}^0 \eta_{n, k}^{(\alpha, \beta)}
  P_{n + k}^{(\alpha + 1, \beta + 1)}  \label{eq:jacform-abp}\\
  &  &  \nonumber\\
  &  & \left. \begin{array}{lll}
    \eta_{n, 0}^{(\alpha, \beta)} & = & 1\\
    &  & \\
    \eta_{n, - 1}^{(\alpha, \beta)} & = & \frac{2 n (\alpha - \beta)}{(2 n +
    \alpha + \beta) (2 n + \alpha + \beta + 2)}\\
    &  & \\
    \eta_{n, - 2}^{(\alpha, \beta)} & = & \frac{- 4 n (n - 1) (n + \alpha) (n
    + \beta)}{(2 n + \alpha + \beta - 1) (2 n + \alpha + \beta)^2 (2 n +
    \alpha + \beta + 1)} = \frac{h_n^{(\alpha, \beta)}}{h_{n - 2}^{(\alpha +
    1, \beta + 1)}}
  \end{array}  \right\}  \label{eq:jacform-etadef}\\
  &  &  \nonumber\\
  \tilde{P}_n^{(\alpha, \beta)} & = & \sum_{k = - 2}^0 \tilde{\eta}^{(\alpha,
  \beta)}_{k, n} P_{n + k}^{(\alpha + 1, \beta + 1)} 
  \label{eq:jacform-abpn}\\
  &  &  \nonumber\\
  &  & \left. \begin{array}{lll}
    \tilde{\eta}_{n, 0}^{(\alpha, \beta)} & = & \sqrt{\frac{h_n^{(\alpha + 1,
    \beta + 1)}}{h_n^{(\alpha, \beta)}}}\\
    &  & \\
    & = & \sqrt{\frac{4 (n + \alpha + 1) (n + \beta + 1) (n + \alpha + \beta
    + 1) (n + \alpha + \beta + 2)}{(2 n + \alpha + \beta + 1) (2 n + \alpha +
    \beta + 2)^2 (2 n + \alpha + \beta + 3)}}\\
    &  & \\
    \tilde{\eta}_{n, - 1}^{(\alpha, \beta)} & = & \eta_{n, - 1}^{(\alpha,
    \beta)}  \sqrt{\frac{h_{n - 1}^{(\alpha + 1, \beta + 1)}}{h_n^{(\alpha,
    \beta)}}} = \frac{2 (\alpha - \beta) \sqrt{n (n + \alpha + \beta + 1)}}{(2
    n + \alpha + \beta) (2 n + \alpha + \beta + 2)}\\
    &  & \\
    \tilde{\eta}_{n, - 2}^{(\alpha, \beta)} & = & \eta_{n, - 2}^{(\alpha,
    \beta)}  \sqrt{\frac{h_{n - 2}^{(\alpha + 1, \beta + 1)}}{h_n^{(\alpha,
    \beta)}}} = - \sqrt{- \eta_{n, - 2}^{(\alpha, \beta)}}\\
    &  & \\
    & = & - \sqrt{\frac{4 n (n - 1) (n + \alpha) (n + \beta)}{(2 n + \alpha +
    \beta - 1) (2 n + \alpha + \beta)^2 (2 n + \alpha + \beta + 1)}}
  \end{array} \right\}  \label{eq:jacform-etatildedef}
\end{eqnarray}
\begin{equation}
  \label{eq:jac-dpromote} \begin{array}{lll}
    \frac{\mathd}{\mathd r} \tilde{P}_n^{(\alpha, \beta)} & = & \sqrt{n (n +
    \alpha + \beta + 1)} \tilde{P}_{n - 1}^{(\alpha + 1, \beta + 1)} =
    \tilde{\zeta}_n^{(\alpha, \beta)} \tilde{P}_{n - 1}^{(\alpha + 1, \beta +
    1)}
  \end{array}
\end{equation}
\begin{eqnarray}
  (1 - r^2) \frac{\mathd}{\mathd r} P_n^{(\alpha, \beta)} & = & (1 - r^2)
  nP_{n - 1}^{(\alpha + 1, \beta + 1)} \nonumber\\
  &  &  \nonumber\\
  & = & n \sum_{k = 0}^2 \varepsilon_{n - 1, k}^{(\alpha + 1, \beta + 1)}
  P_{n - 1 + k}^{(\alpha, \beta)}  \label{eq:jacform-d}\\
  &  &  \nonumber\\
  (1 - r^2) \frac{\mathd}{\mathd r} \tilde{P}_n^{(\alpha, \beta)} & = & (1 -
  r^2) n \sqrt{\frac{h_{n - 1}^{(\alpha + 1, \beta + 1)}}{h_n^{(\alpha,
  \beta)}}} \tilde{P}_{n - 1}^{(\alpha + 1, \beta + 1)} \nonumber\\
  &  &  \nonumber\\
  & = & (1 - r^2) \sqrt{n (n + \alpha + \beta + 1)}  \tilde{P}^{(\alpha + 1,
  \beta + 1)}_{n - 1} \nonumber\\
  &  &  \nonumber\\
  & = & \sqrt{n (n + \alpha + \beta + 1)} \sum_{k = 0}^2 
  \tilde{\varepsilon}_{n - 1, k}^{(\alpha + 1, \beta + 1)} 
  \tilde{P}^{(\alpha, \beta)}_{n - 1 + k}  \label{eq:jacform-dn}
\end{eqnarray}




\section{Hermite polynomials}

\label{app:polynomials-hermite}Another class of polynomials we discuss
are the Hermite polynomials. The monic Hermite polynomials $H^{(0)}_n (x)$, $x
\in \mathbbm{R}$ satisfy
\begin{equation}
  \label{eq:hermitepoly0-ode} \frac{\mathd^2}{\mathd x^2} H_n^{(0)} - 2 x
  \nonesep \frac{\mathd}{\mathd x} H^{(0)}_n + 2 n \nonesep H^{(0)} = 0
\end{equation}
In Sturm-Liouville form this reads
\begin{equation}
  \label{eq:hermitepoly0-slode} \frac{\mathd}{\mathd x} \left( e^{- x^2} 
  \frac{\mathd}{\mathd x} H^{(0)}_n \right) + 2 ne^{- x^2} H^{(0)}_n = 0.
\end{equation}
From equation (\ref{eq:hermitepoly0-slode}) one can read off that the Hermite
polynomials are orthogonal under the weight $w_H^{(0)} (x) = e^{- x^2}$ with
eigenvalues $\lambda_n = 2 n$. The \tmtextit{generalized} Hermite polynomials
are orthogonal under the weight $w_H^{(\alpha)} = |x|^{2 \alpha} e^{- x^2}$
for $\alpha > - \frac{1}{2}$. The recurrence formulae for equations
(\ref{eq:opoly3term-monic}) and (\ref{eq:opoly3term-normal}) are
\[ \begin{array}{lll}
     a_n^{(\alpha)} & = & 0\\
     &  & \\
     b_0^{(\alpha)} & = & \Gamma \left( \alpha + \frac{1}{2} \right)\\
     &  & \\
     b_n^{(\alpha)} & = & \left\{ \begin{array}{lll}
       \frac{n}{2}, &  & n \text{ even}\\
       &  & \\
       \frac{n}{2} + \alpha, &  & n \text{ odd}
     \end{array} \right.
   \end{array} \]
One of the nice properties of the Hermite polynomials is the convenient expression
for differentiation:
\[ \frac{\mathd}{\mathd x} H_n^{(0)} (x) = nH_{n - 1}^{(0)} (x), \hspace{1cm} 
   \text{(monic).} \]
The Hermite functions can be defined as
\[ h_n^{(0)} (x) = e^{- x^2 / 2} H^{(0)}_n (x), \]
where $H_n^{(0)}$ can be either the monic or $L^2$-normalized polynomials and
they satisfy the ODE
\begin{equation}
  \label{eq:hermitefun-slode} \frac{\mathd^2}{\mathd x^2} h^{(0)}_n +
  h_n^{(0)} (2 n + 1 - x^2) = 0.
\end{equation}
The differentiation property for the monic polynomials can be adapted to form
a sparse differentiation representation for the $L^2$-normalized Hermite
functions:
\[ \frac{\mathd}{\mathd x} h_n^{(0)} (x) = \sqrt{\frac{n}{2}} h_{n - 1}^{(0)}
   - \sqrt{\frac{n + 1}{2}} h_{n + 1}^{(0)} . \]


\section{Laguerre polynomials}

\label{app:polynomials-laguerre}Laguerre polynomials are polynomial solutions
to the generalized Laguerre equation
\[ xy'' + (\alpha + 1 - x) y' + \lambda_n y = 0, \hspace{1cm} x \in [0,
   \infty) \]
In Sturm-Louville form this reads
\[ \left[ x^{\alpha + 1} e^{- x} y' \right]' + \lambda_n x^{\alpha} e^{- x} y
   = 0, \hspace{1cm} x \in [0, \infty) \]
When the eigenvalue $\lambda_n = n$ and $\alpha > - 1$, Laguerre polynomials
$L_n^{(\alpha)} (x)$ are solutions to the above differential equations.



The Laguerre polynomials are usually normalized so that
\[ L_n^{(\alpha)} (0) = \left( \begin{array}{c}
     n + \alpha\\
     n
   \end{array} \right) . \]
Given the normalization, the three-term recurrence is
\[ (n + 1) L_{n + 1}^{(\alpha)} = (2 n + \alpha + 1 - x) L_n^{(\alpha)} - (n
   + \alpha) L_{n - 1}^{(\alpha)} . \]
The recurrence coefficients for (\ref{eq:opoly3term-monic}) and
(\ref{eq:opoly3term-normal}) are
\[ \begin{array}{lllll}
     a^{(\alpha)}_n & = & 2 k + \alpha + 1, &  & \\
     &  &  &  & \\
     b^{(\alpha)}_0 & = & \Gamma (1 + \alpha), &  & \\
     &  &  &  & \\
     b^{(\alpha)}_n & = & n (n + \alpha), &  & n > 0.
   \end{array} \]
To parallel the Jacobi case, the Laguerre polynomials also admit
promotion/demotion-type formulae:
\[ \begin{array}{llll}
     L_n^{(\alpha)} & = & L_n^{(\alpha + 1)} - L_{n - 1}^{(\alpha + 1)}, & n >
     0\\
     &  &  & \\
     xL_n^{(\alpha)} & = & (n + \alpha) L_n^{(\alpha - 1)} - (n + 1) L_{n +
     1}^{(\alpha - 1)}, & \alpha > 0\\
     &  &  & \\
     \frac{\mathd}{\mathd x} L_n^{(\alpha)} & = & - L_{n - 1}^{(\alpha + 1)} &
     
   \end{array} \]
